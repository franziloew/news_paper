\documentclass[12pt,a4paper,notitlepage]{article}
\usepackage[utf8]{inputenc}
\usepackage[english]{babel}
\usepackage[T1]{fontenc}
\usepackage[affil-it]{authblk}
\usepackage[backend=biber,
			style=authoryear-icomp,
			sortlocale=de_DE,
			natbib=true,
			isbn=false,
			doi=false,
			bibstyle=authoryear,
			]{biblatex}
\usepackage{eurosym}
\usepackage{enumitem}
\usepackage{url}
\usepackage{blindtext}
\usepackage{hyperref}
\usepackage{breakurl}
\usepackage{amsmath}
\usepackage{titling}
\usepackage{amsfonts}
\usepackage{amssymb}
\usepackage{pgfplots}
\usepackage{caption}
\usepackage{subcaption}
\usepackage{graphicx}
\usepackage{dcolumn}
\usepackage{tikz-3dplot}
\usepackage{subcaption}
\usepackage{float}
\usepackage{adjustbox}
\usepackage{multirow,rotating}
\usepackage[autostyle]{csquotes}
\usepackage[toc,page]{appendix}
\usepackage{lscape}
\usepackage{todonotes}
\usepackage{booktabs}
\usepackage{multirow}
\usepackage{bm}
\usepackage{eurosym}
\usepackage{pdflscape}
\usepackage{geometry}
\geometry{a4paper,left=30mm,right=20mm, top=2cm, bottom=2cm} 

\addbibresource{Textmining.bib}
\ExecuteBibliographyOptions{maxcitenames=2,mincitenames=1}
\renewcommand*{\mkbibnamelast}[1]{\textsc{#1}}

\title{Measuring different types of media bias in German online news articles.}
\date{\today}

\author{Franziska Löw
  \thanks{Electronic address: \texttt{loewf@hsu-hh.de}}}
\affil{Department of Industrial Economics,\\ Helmut Schmidt University,\\ Hamburg, Germany}

\begin{document}
\begin{titlepage}
	\maketitle
	\begin{abstract}
	However, the concept of media bias actually encompasses different subtypes: Visibility bias is the salience of political actors, tonality bias the evaluation of these actors, and agenda bias the extent to which parties address preferred issues in media coverage. Most literature on media bias focuses on only one type of bias, mainly disregarding agenda bias, as the operationalization is somewhat challenging. Additionally, studies that analyse the effect of media content primarily employ manual content analysis and are therefore much more time-consuming and susceptible to errors. Using automated data mining tools this paper provides an approach that allows an efficient and objective analysis to measure the different types of media bias. In order to investigate whether the media landscape transmits a biased reality of the political landscape, online news articles are analysed together with party press releases.
	\end{abstract}

\end{titlepage}

\tableofcontents

\pagebreak

%------------------------------------------------%
%----------------- Begin paper -------------------%
%------------------------------------------------%

\section{Introduction}

What drives media slant?
\citet{gentzkow_what_2010} : Readers have an economically significant preference for like-minded news. Firms respond strongly to consumer preferences, which account for roughly 20 percent of the variation in measured slant in our sample.
- To measure news slant, they examine the set of all phrases used by members of Congress in the 2005 Congressional Record, and identify those that are used much more frequently by one party than by another. We then index newspapers by the extent to which the use of politically charged phrases in their news coverage resembles the use of the same phrases in the speech of a congressional Democrat or Republican.

News content has a powerful impact on politics, with ideologically diverse content producing socially desirable outcomes.

- news content can have significant effects on political attitudes and outcomes. documented empirically by \citet{stromberg_radios_2004, gentzkow_media_2004, gentzkow_television_2006, gerber_does_2009, dellavigna_fox_2007}, and others.

- what are the incentives for ideological content? why do media report more on on party, than on others?

\citet{groseclose_measure_2005} made an important contribution, proposing a new measure of ideological content based on counts of think-tank citations. However, their index was calculated only for a small number of outlets, and has not been used to analyze the determinants of slant.

\citet{gentzkow_what_2010} compute their slant index for a large sample of U.S. daily newspapers. They use their measure to estimate a model of newspaper demand, in which a consumer's utility from reading a newspaper depends on the match between the newspaper's slant and the consumer's own ideology.

The importance of the internet as a source of information for political topics has grown strongly in recent years. Even though television remains the most widely used source of news in Germany (2018: 74\%), numbers watching continue to decline while use of the internet for news has grown significantly in the last year (+5\%, 2018: 65\%) \citep{holig_reuters_2018}. The expansion of the internet as a new method of communication provides a potential challenge to the primacy of the traditional media and political parties as formers of public opinion \citep{savigny_public_2002}.

However, the discussion about the influence of media on the political opinion-forming process has not only been studied in the literature since the growing importance of the internet. The general hypothesis is that the media reporting about political news may have a profound influence on voter opinions and preferences \citep{ferree_four_2002, mccombs_look_2005, eberl_one_2017}. It can therefore be argued, that one central responsibility of the media is to supply voters with balanced and objective information on relevant political issues and actors \citep{stromback_four_2008, eberl_one_2017}. 

When is media biased? ...

Several authors have attempted to measure media bias statistically. \citet{dalessio_media_2000} review studies on media bias in the communications literature, finding little robust evidence of aggregate bias.

According to \citet{dalessio_media_2000} the concept of media bias actually encompasses different subtypes in the communications literature: (1) Coverage bias, (2) tonality bias und (3) agenda bias. These three concepts measure how often political actors appear in the media (coverage bias), how they are evaluated (tonality bias) and whether they are able to present their own political positions and talk about their issues in the media (agenda bias). Most of the literature on media bias focuses on one type of bias and most research tends to disregard agenda bias as the operationalization is more challenging \citet{eberl_one_2017}. In order to know which news stories have been held out by the media, one would have to know the universe of news stories at a given point in time \citep{dalessio_media_2000}. 

% --- Choosing a reference point is crucial ...


1. Endogenous reference point (compared to other parties)

2. Exogenous reference point (poll value, language used by politicans (most frequent phrases), economics indicators,...)

\citet{lott_is_2014}

Lott-Hassett To define unbiased, it constructs a baseline that can vary with exogenous factors (the state if the economy). In contrast, some studies define unbiased simply as some sort of version of "presenting both sides of the story.". Depending on the context, this can be misleading: To see why the latter notion is inappropriate, suppose that a newspaper devoted just as many stories describing the economy under President Clinton as good as it did describing the economy as bad. By the latter notion this newspaper is unbiased. However, by Lott and Hassett's notion the newspaper is unbiased only if the economy under Clinton was average. If instead it was better than average, then Lott and Hassett (as many would recognize as appropriate, including us) would judge the newspaper to have a conservative bias. 

\citet{groseclose_measure_2005} notion of bias also varies with exogenous factors. They measure media bias by estimating ideological scores for several major media outlets. To compute this, we count the times that a particular media outlet cites various think tanks and policy groups, and then compare this with the times that members of Congress cite the same groups. (The Quarterly Journal of Economics)

\citet{gentzkow_what_2010} use the most frequent phrases used by members of Congress to estimate newspaper slant / compare newspapers to one another, though not to a benchmark of "true" or "unbiased" reporting.  The slant index measures the frequency with which 429 different newspapers between 2000-2005 use 2 or 3-word phrases used by members of Congress in the 2005 Congressional Record (and identify those that are used much more frequently by one party than by another). They then index newspapers by the extent to which the use of politically charged phrases in their news coverage resembles the use of the same phrases in the speech of a congressional Democrat or Republican.

I adopt the approach used in \citet{eberl_one_2017}, in using parties' campaign communication as an approximation of the potential universe of news stories. To identify the underlying topics in the text corpus a structural topic model is applied. To measure coverage and tonality bias computational text-mining approaches are applied. 

The data analyzed in this study contains nearly 12.000 online news articles from seven major news provider dated from June 1, 2017 to March 1, 2018 as well as over 1.900 press releases of the parties in the german "Bundestag". As the German federal elections took place on 24th of September 2017 and the formation of the government has taken up a period of about five months, the articles considered inform their readers about both the election promises of the parties (before the election) and the coalition talks (after the election). 

While coverage and tonality can be calculated using simple counting methods, estimating the agenda requires more sophisticated approaches, since the topics covered have to be identified.

To discover the latent topics in the corpus of text data, the structural topic model (STM) developed by \citet{roberts_model_2016} is applied. The STM is an unsupervised machine learning approach that models topics as multinomial distributions of words and documents (as a synonym for news articles) as multinomial distributions of topics, allowing the incorporation of external variables that affect both, topical content and topical prevalence. The results of the generative process of the STM are two posterior distributions: One for the topic prevalence in a document (what is the article or press release about?) and one for the content of a topic (what is the topic about?). In the next step the topics addressed in campaign communication (i.e., the party agenda) are compared with the topics the parties address in media coverage (i.e., the mediated party agenda).

% --- key contributions --- %

this study makes two key contributions. (1) I use a combination of different text-mining techniques to analyze a large text corpus that allows an extensive content analyses of newspaper coverage and party press releases and at the same time reduces human induced bias and makes research more traceable and comparable. (2) I present a new approach to measure agenda bias by combining approaches from political and communication sciences with current text analysis techniques.    

% --- remaining course of the paper... --- %

The remaining course of the paper is as follows: The following section provides an overview of the related literature. Section \ref{ch_elections} gives a introduction to the political trends in the past six month (June 2017 to March 2018). The data used to conduct the model is described in section \ref{ch_data}. Section \ref{ch_model} explains the generative process of the structural topic model as well as the selected parameters to run the model. The empirical analysis is conducted in section \ref{ch_empirical}. 

%-------------------------------------------------------%
%----------------- Literature Review -------------------%
%-------------------------------------------------------%

\section{Literature}

The general hypothesis of the literature on the influence of media bias on on the political opinion-forming process is that a biased media reporting in political news may have a profound influence on voter opinions and preferences \citep{ferree_four_2002, mccombs_look_2005, eberl_one_2017}. In the literature on the influence of media bias on the political opinion-forming process usually encompasses different subtypes of media bias \citep{eberl_one_2017, dalessio_media_2000}: (1) Visibility bias, (2) tonality bias and (3) agenda bias. These three concepts measure how often political actors appear in the media (visibility bias), how they are evaluated (tonality bias) and whether they are able to present their own political positions and talk about their issues in the media (agenda bias).
  
% ---- visibility bias ----
There is visibility bias when a party is the subject of an undue amount of coverage compared to the benchmark of that party at a given point in time. Media that reports biased in that sense influences voters behaviour in such way, that voters tend to prefer parties that are more visible in their media repertoire \citep{eberl_one_2017}. Studies combining media content data with voter surveys have indeed found that the mere visibility of parties and candidates is an important factor influencing vote choice (Oegema & Kleinnijenhuis, 2000). The amount of a parties campaign communication or their standing in polls are commonly used as a reference point \citep{junque_de_fortuny_media_2012, hopmann_political_2012}. However, these benchmarks do not allow for a comparison between media outlets. \citet{eberl_one_2017} use the average visibility of all parties in each media outlet during the period of their analysis as a key benchmark to capture whether party visibility is biased in comparison to what is typical for that outlet and are therefore able to compare party visibility between outlets. Applying a similar logic to tonality and agenda bias, they measure the effect of the different bias on user voting behaviour using an online panel survey from the Austrian parliamentary election campaign of 2013. Other empirical studies that measure the influence of media visibility on the electoral behaviour of voters are usually based on regional differences in the reach of certain media \citep{enikolopov_media_2011, dellavigna_fox_2006, snyder_press_2010}. \citet{dewenter_can_2018} use human-coded data from leading media in Germany together with the German Politbarometer survey to investigate how media coverage affects short- and long-term political preferences between February 1998 and December 2012. They find a positive correlation between the media coverage and the short-term voting intention for a political party. In the long-term, however, voting preferences are stable. However, in \citet{eberl_one_2017} the coverage bias effect can not be confirmed, as the effect of visibility bias is positive but not statistically significant.

% ---- tonality bias ----
In addition to visibility, the tonality of the reporting is also important, as it provides consumers with a certain understanding of politics. For example, \citet{druckman_impact_2005} argue that the audience's conclusions about parties are automatically drawn from positive or negative descriptions in texts about the parties. Similarly, valence framing suggests that public awareness of parties is affected depending on whether they are highlighted with positive or negative aspects in the media \citep{de_vreese_valenced_2006, hurtikova_importance_2017}.  

To measure tonality in a text, studies differ between manually coded data (e.g. \citet{eberl_one_2017}) and dictionary-based analysis ( e.g. \citep{junque_de_fortuny_media_2012}). The latter approach is widely used outside the area of media content analysis. To conduct such an analysis, a lists of words (dictionary) associated with a given emotion, such as negativity is pre-defined by the analyst. The document is then deconstructed into individual words and the frequencies of words contained in a given dictionary are calculated. Such lexical or "bag-of-words"€ approaches are widely presented in the finance literature to determine the effect of central banks' monetary policy communications on asset prices and real variables (\citet{nyman_news_2018} \citet{tetlock_giving_2007}, \citet{tetlock_more_2008}). \citet{hansen_shocking_2016} use a similar approach to explore the effects of FOMC (Federal Open Market Committee) statements on both market and real economic variables. To calculate their score, they subtract the negative words from the positive words und divide this by the number of total words of the statement. A similar score is used by \citet{nyman_news_2018}, who measure the effect of narratives and sentiment of financial market text-based data on developments in the financial system. 

In the domain of media content analysis \citet{junque_de_fortuny_media_2012} count the sentiment words in a window of two sentences before and after the mention of a political party and assuming uniformity of sentiment distribution among parties to measure the bias.\footnote{A similar approach for target identification with a 10-word window is used in \citet{balahur_sentiment_2013}} They use a text-mining approach to automate the analysis of a large text corpus showing techniques to measure both visibility and tonality bias. The former is benchmarked by the amount of preference votes for that party. Similar to \citet{junque_de_fortuny_media_2012} the present study uses techniques that allow the computational analysis of a large dataset of text-data. However, a different reference point is used to allow for a comparison between media outlets. Additionally, agenda bias is measured based on the comparison between the content of online news and parties press releases \citet{eberl_one_2017}.  

% ---- agenda bias ----
Overall, most research tends to disregard agenda bias as the operationalization is more challenging. In order to know which news stories have been held out by journalists, the true universe of news stories at a given point in time has to be known \citet{dalessio_media_2000}. However, a greater dissemination of a party's political content may have a positive impact on attitudes towards that party \citep{benewick_floating_1969, eberl_one_2017}. \citet{brandenburg_party_2006} measure partisan tendencies in reporting in terms of all three biases. Utilizing content analysis data from the 2005 General Election campaign they show that increasingly ambiguous endorsements translate into an absence of open support for political parties. Similarly, \citet{eberl_one_2017} find that voters evaluate parties more favorably if those parties addressed their own favored topics more prominently in media coverage. In their analysis media content was analyzed using manual content analysis of political claims on a sentence level.

% -----  STM --------
The STM developed by \citet{roberts_model_2016} is a recent extension of the standard topic modelling technique, labeled as latent Dirichlet allocation (LDA), which refers to the Bayesian model in \citet{blei_latent_2003} that treats each word in a topic and each topic in a document as generated from a Dirichlet - distributed prior.\footnote{See also \citet{griffiths_probabilistic_2002}, \citet{griffiths_finding_2004} and \citet{hofmann_probabilistic_1999}. \citet{pritchard_inference_2000} introduced the same model in genetics for factorizing gene expression as a function of latent populations.} Since its introduction into text analysis, LDA has become hugely popular and especially useful in political science.\footnote{see \citet{blei_probabilistic_2012}, \citet{grimmer_text_2013} and \citet{wiedmann_text_2016} for an overview in social science and \citet{gentzkow_text_2017} give an overview of text mining applications in economics.} \citet{wiedmann_text_2016} uses topic model methods on large amounts of news articles from two german newspapers published between 1959 and 2011, to reveal how democratic demarcation was performed in Germany over the past six decades. \citet{paul_cross-collection_2009} compares editorial differences between media sources, using cross-collection latent Dirichlet allocation (ccLDA), an LDA-based approach that incorporates differences in document metadata. They use a dataset of 623 news articles from August 2008 from two American media outlets - msnbc.com and foxnews.com - to compare how they discuss topics. Reviewing the top words of the word-topic distribution, they find some content differences between the two media sources under review. 

The difference between the widely used LDA and the STM approaches lies in how $\theta$ and $\phi$ are determined. LDA assumes that $\theta ~ \text{Dirichlet}(\alpha)$ and $\phi ~ \text{Dirichlet}(\beta)$, where $\alpha$ and $\beta$ are fitted with the model. While for STM, the prior distributions for $\theta$ and $\phi$ depend on document-level covariates (e.g. the author or date of a document). For this purpose, the the STM specifies two design matrices of covariates, where each line defines a vector of covariates for a specific document.  In $X$, the covariates for topic prevalence are given, so that the probability of a topic for each document varies according to X, rather than resulting from a single common prior. The same applies to $Z$, in which the covariates for the word distribution within a topic are specified. 

The model has been applied to multiple academic fields: \citet{roberts_structural_2014} uses STM to analyze open-ended responses from surveys and experiments, \citet{farrell_corporate_2016} applies the model to scientific texts on climate change, revealing links between corporate funding and the framing of scientific studies. \citet{mishler_using_2015} show that "STM can be used to detect significant events such as the downing of Malaysia Air Flight 17" when applied to twitter data. Another study shows how STM can be used to explore the main international development topics of countries'€™ annual statements in the UN General Debate and examine the country-specific drivers of international development rhetoric \citep{baturo_what_2017}. \citet{mueller_reading_2016} use newspaper text to predict armed conflicts in different regions. They use the estimated topic shares in linear fixed effects regression to forecast conflict out-of-sample. \citet{roberts_navigating_2016} use STM to examine the role of partisanship in topical coverage using a corpus of 13,246 posts that were written for 6 political blogs during the course of the 2008 U.S. presidential election. With the aim of revealing the effect of partisan membership on topic prevalence, each blog is assigned to be either liberal or conservative. To explore the differences between the two, they look at the expected proportion of topics and examine the posts most associated with a respective topic. This approach is similar to \citet{roberts_model_2016}. 

The present analysis differs from earlier approaches to measure agenda bias in that a machine learning technique is used to identify the underlying topics in the text corpus applying a structural topic model \citep{roberts_model_2016}. Furthermore, I use text-mining techniques to measure coverage and tonality bias. However, I shall refrain as far as possible from interpreting the results at political level. Rather, it is my goal to show how text mining techniques enable an efficient and objective analysis of today's online media landscape and simultaneously allow the analysis to be reproduced. 
 
% -------------------------
% Background Bundestagswahl
% -------------------------
\section{Background on the federal election in Germany (2017)}\label{ch_elections}

The articles analyzed in this paper cover a period from June 1, 2017 to March 1, 2018 and thus cover both the most important election campaign topics for the Bundestag elections on September 24, 2017 and the process of forming a government that lasted until February 2018. After four years in a grand coalition with the Social Democrats (SPD), German Chancellor Angela Merkel, member of the conservative party CDU/CSU (also known as Union), ran for re-election. The SPD nominated Martin Schulz as their candidate. 

On the right side of the political spectrum, AfD (alternative for Germany) managed to be elected to the German Bundestag for the first time in 2017. The political debate about the high refugee numbers of the past years brought a political upswing to the AfD, which used the dissatisfaction of parts of the population to raise its own profile. In the course of the reporting on the federal elections, leading party members of the AfD as well as party supporters repeatedly accused the mass media of reporting unilaterally and intentionally presenting the AfD badly.

After the election, the formation of a government was difficult due to the large number of parties elected to the Bundestag and the considerable loss of votes by the major parties CDU/CSU and SPD. Since all parties rejected a coalition with the AfD, numerically only two coalitions with an absolute parliamentary majority were possible: a grand coalition ("GroKo" - from the German word Groߟe Koalition) of CDU/CSU and SPD, and a Jamaica coalition (coalition of CDU/CSU, FDP (economic liberal party) and B90/Die Grünen (Bündnis 90/Die Grünen, green party)). The grand coalition was initially rejected by the SPD. The four-week exploratory talks on the possible formation of a Jamaica coalition officially failed on November 19, 2017 after the FDP announced its withdrawal from the negotiations. FDP party leader Christian Lindner said that there had been no trust between the parties during the negotiations. The main points of contention were climate and refugee policy. CDU and CSU regretted this result, while B90/Die Grünen sharply criticized the liberals'€™ withdrawal. The then Green leader Cem Özdemir accused the FDP of lacking the will to reach an agreement.

After the failure of the Jamaica coalition talks, a possible re-election or a minority government as alternatives were discussed in the media before the SPD decided to hold coalition talks with the CDU/CSU. This led to great resistance from the party base, which called for a party-internal referendum on a grand coalition. After the party members voted in favor of the grand coalition, a government was formed 171 days after the federal elections. 

Figure \ref{fig_polls} shows that support for the two major popular parties has been declining in recent months since August 2017, with the CDU/CSU again showing positive survey results since November 2017. However, the poll results of the SPD have been falling since March 2017. At the same time, the AfD in particular has been recording increasingly positive survey results since June 2017.  

\begin{figure}[H]
\begin{center}
	\caption{Election Polls}
	\includegraphics[width=0.6\textwidth]{../figs/polls}
	\label{fig_polls}
	\end{center}
\end{figure}

% -----
% Data
% -----
\section{Data}\label{ch_data}

% latex table generated in R 3.5.1 by xtable 1.8-3 package
% Sun Nov 11 21:36:13 2018
\begin{table}[ht]
\centering
\begin{tabular}{lrrr}
  \hline
medium & total\_articles & share & visits \\ 
  \hline
Bild.de & 3828 & 0.28 & 156867922 \\ 
  DIE WELT & 18663 & 0.14 & 52962285 \\ 
  FOCUS Online & 9235 & 0.22 & 68730873 \\ 
  SPIEGEL ONLINE & 5590 & 0.31 & 96139323 \\ 
  stern.de & 15804 & 0.15 & 18192672 \\ 
  tagesschau.de & 2261 & 0.41 & 75400000 \\ 
  ZEIT ONLINE & 5467 & 0.23 & 30600773 \\ 
   \hline
\end{tabular}
\caption{News sources used for the analysis} 
\end{table}


I conduct the estimation on a sample of 11,880 online news articles from seven German news providers about domestic politics\footnote{Bild.de, DIE WELT, FOCUS ONLINE, SPIEGEL ONLINE, stern.de, ZEIT ONLINE, Tagesschau.de}. The articles are dated from June 1, 2017 to March 1, 2018. I first extract all online articles using the Webhose.io API.\footnote{For more information see https://docs.webhose.io/v1.0/docs/getting-started. The scraping code was written in Python and can be made available on request.} Overall, the selected news providers are among the top ten German online news providers - in terms of site visits\footnote{The term visit is used to describe the call to a website by a visitor. The visit begins as soon as a user generates a page impression (PI) within an offer and each additional PI, which the user generates within the offer, belongs to this visit.} - in the period under review, with only Tagesschau.de belonging to the public media. The reason for this is that the content structure of Tagesschau.de is most similar to that of the private providers. Other public media offers provide their content in video (ZDF.de) or audio (Deutschlandfunk (DLF))) format, which make them difficult to compare. In order to limit the analysis to articles on domestic politics, all articles which mention at least one of the major parties\footnote{The exact search terms were "CDU, CSU, Union", "SPD", "FDP", "Grüne", "Linke", "AfD".} have been filtered out. 

 Figure \ref{fig_distr1} shows the distribution of the number of articles by date. There is a high peak around the federal elections on September, 24th and another one shortly after the failure of the Jamaica coalition talks on November, 19th (indicated by the red dotted lines). Figure \ref{fig_distr2} shows that DIE WELT published the most articles on domestic policy, followed by stern.de and FOCUS ONLINE.  

\begin{figure}[H]
	\caption{Article distribution...}
	\begin{center}
		\begin{subfigure}[normla]{0.59\textwidth}
			\includegraphics[width=\textwidth]{../figs/article_timeline.png}
			\caption{...by date}
			\label{fig_distr1}
		\end{subfigure}
		\begin{subfigure}[normla]{0.4\textwidth}
			\includegraphics[width=\textwidth]{../figs/visits.png}
			\caption{... by news provider}
			\label{fig_distr2}
		\end{subfigure}
	\end{center}
\end{figure}

\section{Estimate bias}

1. Estimate different bias types 
2. Estimate a fixed effect regression, to analyze the impact of the different biases that vary over time.

In unbiased media reporting all political sides should be equally represented according to some kind of benchmark for balance or neutrality \citep{hopmann_political_2012}. Bias is then defined as the extent to which media reporting deviates from this benchmark. In this section the measurement methodology for each of the three types of bias defined in the literature (visibility bias, tonality bias and agenda bias \citet{junque_de_fortuny_media_2012, eberl_one_2017}) is described. The strategy for calculating the bias follows the same pattern in all three cases: First, the value for which the bias will be determined is calculated for each party (i.e. visibility, tonality and agenda setting). This is done at a monthly level of the different news providers. The average value of all other parties in the month and the medium serves as the reference value to calculate the bias. To ensure comparability between the different bias metrics, they are standardized to range from −1 to 1, where a party would have a bias of 0 (neutral), when its visibility, tonality or agenda is equal to the mean visibility, tonality or agenda across all parties (in that media outlet) \citep{eberl_one_2017}.

\subsection{Visibility Bias}
 
The overall visibility for each party in a news provider is defined as the number of news articles in which a party is named at least once, normalised on the total amount of articles by that news provider in the corpus. Next, to distinguish between balanced and biased reporting the average visibility of all other parties in a news outlet is used as a key benchmark.\footnote{In another setting \citet{junque_de_fortuny_media_2012} uses the popularity in terms of votes as the a priori fair distribution. The visibility bias of a media source is the difference between the real distribution and the fair distribution.} The results displayed in Figure \ref{fig_visibility_bias} do not show clear evidence for a difference between the different news provider: Overall, they are positively biased towards "Union" and "SPD" and more negatively biased towards the other parties, with "DIE LINKE" having the most negative bias overall. The SPD has the greatest differences over time: The party has a negative bias during the Jamaica negotiations (Sep-Nov), turning positive following the failure of the Jamaica negotiations and the start of the coalition talks on the "GroKo". The opposite is true for "FDP" and "Grüne": Both show a rather positive or neutral bias during these month and a negative bias for the rest of the rime. Visibility of "AfD" is biased negatively during the time of analysis, showing less negativity during the election month (Sep).

\begin{figure}[H]
	\caption{Visibility Bias}
	\begin{center}
		\includegraphics[width=0.8\textwidth]{../figs/visibility_bias.png}
		\label{fig_visibility_bias}
	\end{center}
\end{figure}

% -------------
% sentiment bias 
% --------------

\subsection{Sentiment Analysis}


To measure the tone (or sentiment) of an article a dictionary-based method is applied. To conduct such an analysis, a list of words (dictionary) associated with a given emotion, such as negativity is pre-defined. The document is then deconstructed into individual words and each word is assigned a sentiment value according to the dictionary, where the sum of all values results in the emotional score for the given document. Such lexical or "bag-of-words"€ approaches are widely presented in the finance literature to determine the effect of central banks' monetary policy communications on asset prices and real variables (\citet{nyman_news_2018} \citet{tetlock_giving_2007}, \citet{tetlock_more_2008}). \citet{hansen_shocking_2016} use a similar approach to measure "the two Ts" (Topic and tone). They explore the effects of FOMC (Federal Open Market Committee) statements on both market and real economic variables. To calculate their score, they subtract the negative words from the positive words und divide this by the number of total words of the statement. A similar score is used by \citet{nyman_news_2018}, who measure the effect of narratives and sentiment of financial market text-based data on developments in the financial system. They count the number of occurrences of excitement words and anxiety words and then scale these numbers by the total text size as measured by the number of characters.

The present paper uses a dictionary that lists words associated with positive and negative polarity weighted within the interval of $[-1; 1]$. SentimentWortschatz\footnote{SentiWS for short. available here: http://wortschatz.uni-leipzig.de/de/download}, is a publicly available German-language resource for sentiment analysis, opinion mining, etc.. The current version of SentiWS (v1.8b) contains 1,650 positive and 1,818 negative words, which sum up to 15,649 positive and 15,632 negative words including their inflections, respectively. Table \ref{t_sentdict} shows ten examples entries of the dictionary. To obtain a more reliable correlation between the "target" (a political party) and the word's polarity score, sentiment words are counted in a window of two sentences before and after the mention of a political party \citep{junque_de_fortuny_media_2012}. The tonality score of an article is then calculated from the sum of the these words divided by the total number of words in that article. Again, the tonality bias for is then computed as the deviation of each party's specific tonality from the average tonality of all other parties in that outlet and standardized to range from −1 to 1.

% latex table generated in R 3.4.2 by xtable 1.8-2 package
% Wed May  9 00:35:00 2018
\begin{table}[ht]
\centering
\begin{tabular}{rlr}
  \hline
 & word & value \\ 
  \hline
1 & brisant & -0.00 \\ 
  2 & makel & -0.18 \\ 
  3 & vergöttern & 0.00 \\ 
  4 & gediegen & 0.08 \\ 
  5 & mühe & -0.00 \\ 
  6 & untreue & -0.33 \\ 
  7 & unhöflich & -0.00 \\ 
  8 & erschweren & -0.49 \\ 
  9 & addieren & 0.00 \\ 
  10 & fehlleistung & -0.00 \\ 
   \hline
\end{tabular}
\caption{Sentiment Dictionary (Sample)} 
\label{t_sentdict}
\end{table}

 
\begin{figure}[H]
	\caption{Tonality Bias}
	\begin{center}
		\includegraphics[width=0.8\textwidth]{../figs/tonality_bias.png}
		\label{fig_tonality_bias}
	\end{center}
\end{figure}


% ---------------------
% Agenda bias
% ----------------------
\subsection{Agenda Bias}\label{ch_agendabias}

To allow for an operationalization of agenda bias, parties' press releases are used as an approximation of the potential universe of news stories \citep{eberl_one_2017}. The press releases were scraped from the website of each party\footnote{https://www.afd.de/presse, https://www.spdfraktion.de, https://www.die-linke.de/start/presse/aus-dem-bundestag, https://www.fdp.de, https://www.gruene-bundestag.de/, https://www.presseportal.de/nr/7846} to compare the topics addressed with the policy issues in the news articles. 

\subsubsection{Data preparation}

To use text as data for statistical analysis, different pre-processing steps have to be conducted. In fact, in order to use text as data and reduce the dimensionality to avoid unnecessary computational complexity and overfitting, pre-processsing the text is a central task in text mining \citep{bholat_text_2015}. Intuitively the term frequency (tf) of a word is a measure of how important that word may be for the understanding of the text. As can be seen in Figure \ref{fig_wordcloud1}, problems arise with words that are highly frequent. For example "die", or "der (eng. "the"), "und" (eng. "and"), and "ist" (eng. "is") are extremely common but unrelated to the quantity of interest. These terms, often called stop words \citep{gentzkow_text_2017}, are important to the grammatical structure of a text, but typically don't add any additional meaning and can therefore be neglected. A common strategy to reduce the number of language elements is to pre-process the text by imposing some preliminary restrictions (e.g. stop-word removal and stemming) based on the nature of the data (e.g. twitter text, newspaper articles, speeches, etc.) \citep{gentzkow_text_2017}. 

\begin{figure}[H]\label{fig_wordcloud1}
	\begin{center}
		\begin{subfigure}[normla]{0.7\textwidth}
			\includegraphics[width=\textwidth]{../figs/wordcloud.png}
			\caption{Wordcloud (before pre-processing)}
		\end{subfigure}
	\end{center}
\end{figure}

To remove distorting words, the pre-defined stop word list from the Snowball project\footnote{http://snowball.tartarus.org/algorithms/german/stop.txt} is used together with a customized list of stop-words. Additionally punctuation character (e.g. ., ,, !, ?, etc.) and all numbers are removed from the data. A next step to reduce the dimensionality of text data is to apply an adequate stemming technique. Stemming is a process by which different morphological variants of a word are traced back to their common root. For example, "voting" and "vote" would be treated as two instances of the same token after the stemming process. There are many different techniques for the stemming process. I apply the widely used Porter-Stemmer algorithm, which is based on a set of shortening rules that are applied to a word until it has a minimum number of syllables.\footnote{https://tartarus.org/martin/PorterStemmer/} After completing these steps 68,576 \todo{check, if this is correct.} unique terms were left in the vocabulary. The word clouds in Figure \ref{fig_wordcloud2} represent the most frequent words of the pre-processed articles for Bild.de and Tagesschau.de.\footnote{The wordclouds of the other parties can be found in the appendix \ref{apx_wordclouds}} It becomes evident that these are texts discussing domestic policy issues. The SPD in particular seems to be highly frequent. However, at first glance, there are no obvious differences between the news providers.  

\begin{figure}[H]\label{fig_wordcloud2}
	\caption{Wordclouds}
	\begin{center}
		\begin{subfigure}[normla]{0.48\textwidth}
			\includegraphics[width=\textwidth]{../figs/wordclouds/Bild.de.png}
			\caption{Bild.de (after pre-processing)}
		\end{subfigure}
		\begin{subfigure}[normla]{0.48\textwidth}
			\includegraphics[width=\textwidth]{../figs/wordclouds/tagesschau.de.png}
			\caption{Tagesschau.de (after pre-processing)}
		\end{subfigure}
	\end{center}
\end{figure}

 % --------------------
% Document-term-matrix
% --------------------
The next step is to divide the entire dataset into individual documents and to represent these documents as a finite list of unique terms. In this setting, each news article and each press release represents a document $d$, whereby each of these documents can be assigned to a news website or a party. The sum of all documents forms what is called the corpus. For each document $d \in \lbrace 1,...,D \rbrace$ the number of occurrences of term $v$ in document $d$ is computed, in order to obtain the count $x_{d,v}$, where each unique term in the corpus is indexed by some $v \in \lbrace 1,...,V \rbrace$ and where $V$ is the number of unique terms. The $D$ x $V$ matrix $\boldsymbol{X}$ of all such counts is called the document-term matrix. Each row in this matrix represents a document, where each entry in this row counts the occurrences of a unique term in that document. This representation is often referred to as the bag of words model \citep{gentzkow_text_2017}, since the order in which words are used within a document is disregarded.

\subsubsection{Structural topic model}

To find out the latent structure of each document, a structural topic model (STM) is estimated. In general, topic models formalize the idea that documents are formed by hidden variables (topics) that generate correlations among observed terms. They belong to the group of unsupervised generative models, meaning that the true attributes (topics) cannot be observed. One crucial assumption to be made for such models is the number of topics ($K$) that occur over the entire corpus. 

Each individual topic potentially contains all of the unique terms within the vocabulary $V$ with different probability. Therefore, each topic $k$ can be represented as a probability vector $\phi_k$ over all unique terms $V$. Simultaneously, each individual document $d$ in the corpus can be represented as a probability distribution $\theta_d$ over the $K$ topics. The underlying data generating process to generate each individual word $w_{d,n}$ in a document $d$ for the $n^{th}$ word-position can be described as follows:\footnote{A more detailed description of the generative process of the STM can be found in section \ref{ch_generativeProcess}}

\begin{enumerate}
	\item for each document $i$, draw its distribution of topics $\theta_d$ depending on the metadata included in the model; 
	\item for each topic $k$, draw its distribution of words $\phi_k$ depending on the metadata included in the model;
	\item for each word $n$, draw its topic $z_n$ based on $\theta_i$;
	\item for each word word $n$, draw the term distribution for the selected topic $\phi_{z_{d,n}}$.
\end{enumerate}

% ----------------------
% Model Selection
% ----------------------
Inference of mixed-membership models, such as the one applied in this paper, has been a thread of research in applied statistics in the past few years (\citet{blei_latent_2003} \citet{erosheva_mixed-membership_2004} \citet{braun_variational_2010}). Topic models are usually imprecise as the function to be optimized has multiple modes, such that the model results can be sensitive to the starting values (e.g. the number of topics). Since an ex ante valuation of a model is hardly possible, I compute a variety of different models and compare their posterior probability. This enables me to check how results vary for different model solution \citep{roberts_navigating_2016}. I then cross-checked some subset of assigned topic distributions to evaluate whether the estimates align with the concept of interest \citep{gentzkow_text_2017}. These manual audits are applied together with numeric optimization based on the topic coherence measure suggested by \citet{mimno_optimizing_2011}. 

\todo{describe model specifications}
This process revealed that a model with 55 topics best reflects the structure in the corpus. Furthermore, the source (news website or party) of a document is used as covariate in the topic prevalence. In other words, the corresponding news website or party of an article or press release influences the probability distribution of topics for that document. Additionally I assume that the topical content differs between news articles and press releases. 

\subsubsection{Results}
% ---- Results ---- %
In order to get an initial overview of the results, Figure \ref{fig_expected_freq} displays the topics ordered by their expected frequency across the corpus. To assign a label to each topic, I looked at the most frequent words in that topic and the most representative articles \citep{roberts_model_2016}. 

It becomes apparent that topic 4 about the coalition talks between CDU/CSU and SPD - the "Grand coalition" or "GroKo" - is the topic with the highest expected frequency in the whole corpus, followed by the topic about the so-called Jamaica parties (CDU/CSU, FDP and B90/Die Grünen), which was the first alternative to be negotiated directly after the elections.  

\begin{figure}[H]\label{fig_wordcloud2}
	\caption{Expected frequency}
	\begin{center}
		\begin{subfigure}[normla]{0.48\textwidth}
			\includegraphics[width=\textwidth]{../figs/topic_proportion_news.png}
			\caption{Online news}
		\end{subfigure}
		\begin{subfigure}[normla]{0.48\textwidth}
			\includegraphics[width=\textwidth]{../figs/wordclouds/topic_proportion_press.png}
			\caption{Press releases}
		\end{subfigure}
	\end{center}
\end{figure}


\section{Fixed effects regression}\label{ch_regression}

This section seeks to examine the association between sentiment reflected in online news content and phone poll results in Germany. Specifically, it aims to find the extent to which online sentiment and phone survey results correlate given a number of lags. I use the data from the "Sonntagsumfrage" (Sunday survey) from infratest dimap.\footnote{https://www.infratest-dimap.de/umfragen-analysen/bundesweit/sonntagsfrage/} The institution regularly asks at least 1000 German citizens the question: "Which party would you choose if federal elections were to take place next Sunday?" The survey thus measures the current election tendencies and therefore reflects an intermediate state in the opinion-forming process of the electoral population.

Much of the research on online content and political trends have focused on traditional weblogs and social media websites, such as Twitter, Facebook, MySpace, and YouTube. These studies have shown that social media is used to spread political opinions and that these considerations reflect the political landscape of the offline world. \citet{tumasjan_predicting_2010} investigate Tweets between August 13th and September 19th, 2009, prior to the German national elections to examine whether Twitter messages reflect the current offline political sentiment and whether it can be used to predict the popularity of parties or coalitions in the real world. With regard to the later question, they compare the share of attention the political parties receive on Twitter with the election result to examine whether the activity on Twitter can serve as a predictor of the election outcome. They found that the number of tweets reflects the election result and even comes close to traditional election polls.

\citet{fu_analyzing_2013} use a corpus of online posts from discussion forums and blogs to examine the extent to which online sentiment reflected in social media content can predict phone survey results in Hong Kong. They build a sentiment classifier conducting a support vector machine analysis on a training set of 2,000 manually labeled posts. In order to evaluate the temporal relationship between the time series of the online sentiment score and the results of the telephone survey, a cross correlation analysis was conducted, using the Box and Jenkins autoregressive integrated moving average (ARIMA) method \citep{box_time_2008}. Estimating the cross-correlation functions of the residuals, they find that online sentiment scores can lead phone survey results by about 8–15 days. 

In a more recent conference paper, \citet{padmaja_evaluating_2014} identify the scope of negation in news articles for two political parties in India (BJP and UPA) to analyze how the choice of certain words used in these texts influence the sentiments of public in polls. Comparing three different sentiment analysis methods (two machine learning and one dictionary method), they observe that the choice of certain words used in political text was influencing the sentiments in favor of BJP. They conclude that this sentiment bias might be one of the causes for the election results in 2014.

\citet{dewenter_can_2018} use human-coded data from leading media in Germany together with the German Politbarometer survey to investigate how media coverage affects short- and long-term political preferences between February 1998 and December 2012. They find a positive correlation between the media coverage and the short-term voting intention for a political party. In the long-term, however, voting preferences are stable. 

In the present paper, the relationship between monthly average of both the sentiment value of individual topics ($x_t$) and the survey value of the parties ($y_t$) is estimated using the cross correlation function (CCF). Thus, the CCF between $x_{t+h}$ and $y_t$ for $h\pm 1$,$h \pm 2$,$h \pm 3$ is computed. A negative value for $h$ is a correlation between the topic sentiment value at a time before $t$ and the survey value at time $t$. The correlation value for $h=0$ indicates the contemporary correlation between the two time series.  Based on the coefficients of the cross correlation estimation shown in Figure \ref{fig_ccf}, the significant correlations between topic sentiment and survey value are evaluated for each party.\footnote{The value of the cross correlation coefficients for lag 0 are listed in the appendix \ref{apx_ccf}} It is important to note that no causal relationships are described below, but only the correlation between the two time series. 

The survey results of the AfD correlate negatively with topics relating to the SPD (17, 22) at lag 0. Thus, if the SPD was more negatively reported, the poll value of the AfD increased in the same month (and vice versa). Another significant negative correlation exists between the reporting on the GroKo (4) and the survey value of AfD at lag -1 ($x_{t-1}$). So if the GroKo was more negatively reported in one month, the survey value of the AfD increased in the following month (and vice versa). For the FDP, too, only negative correlation coefficients can be detected, with the strongest negative correlation existing for the topic relating to the CSU (23). If the CSU got off worse in the online news, the poll value of the FDP went up. Another interesting observation is that the FDP's poll results correlate negatively with issues relating to the Jamaica coalition at lag 1 ($x_{t+1}$). So if the poll results for the FDP rose in one month, the following month the FDP was reported more negatively. The Green Party survey results show no negative correlation with any of the topics, except topic 30 at lag 1. It is striking that there seems to be a strong negative correlation between the SPD topics (1, 17, 22) and the poll results of the left party (DIE LINKE). This means that the poll value of the left party has climbed if the topics related to the SPD were discussed more negatively. Same applies to the reporting on the GroKo (30) for lag -1. Conversely, the SPD's survey results correlate strongly positively with these topics, and also with topic 30 with a delay of one month. For the CDU/CSU, too, only significant negative correlations are discernible: the survey results correlate negatively with the topic of the Schulz v Merkel debate (10) and negatively with topic 30 with a delay of one month ($x_{t+1}$). 

\begin{landscape}
\begin{figure}[H]
	\caption{Cross-Correlation Coefficients}
	\begin{center}
			\includegraphics[width=1.3\textwidth,keepaspectratio]{figs/ccf2.png}
			\label{fig_ccf}
	\end{center}
\end{figure} 
\end{landscape}

After the figures above have been analyzed, the following points can be summarized:

\begin{enumerate}
	\item Only the survey results of the SPD correlate positively with the emotional value of the topics. There seems to be a strong correlation between the way topics concerning the SPD are discussed in the online news and the poll results.  
	\item The poll results of the Left Party, on the other hand, seem to correlate negatively with the reporting on the SPD. 
	\item Similar tendencies can also be seen with regard to the AfD, since here too the survey results correlate significantly negatively with the topics about the SPD and the grand coalition. 
\end{enumerate} 

Summarizing the analyses from this and the previous section, it can be observed that the positive correlation between the emotional value of the reporting and the survey value of a party is particularly large if the reporting is conspicuously negative. 

\section{Conclusion}

The ongoing discussion about the influence of digital media on the political opinion-forming process addresses the question whether there are convergence tendencies within the mass media and whether the reporting in the media correlates with the voting preferences. To analyze this question, this paper examines (1) whether the political reporting of different media differs in terms of topic frequency and topic tonality and (2) whether the emotional value of the reporting correlates with poll results. 

Using text data of 14,937 online news articles from seven German news providers about domestic politics, I first estimate a Structural Topic Model to find the latent topics in the news articles in order to answer (1). After assigning a topic to each news article, the sentiment value of articles about contemporary political events is calculated using a dictionary-based method. In order to tackle (2), the results from the sentiment analysis are then compared to poll results.

Regarding (1), the analysis revealed that there are differences between the media considered, both in terms of topic prevalence and the way in which these topics are discussed. Although the topics are discussed negatively on average, differences can still be observed, especially regarding topics that deal with the coalition negotiations. The smallest differences were observed for topics concerning the AfD. However, no evidence has been found that the media systematically report more negatively on the AfD than on other parties. With regard to (2), the analysis has shown that the tonality of topics discussed by the SPD shows a strong positive correlation to current survey results. Overall, there seems to be a link between reporting on political issues and electoral preferences. The results of this study show evidence that the content of media could have an influence on the opinion-forming process of the voters and therefore underline the responsibility of media in the political context.

\pagebreak

\printbibliography

\appendix
\section{Appendices}

\subsection{Generative Process of STM}\label{ch_generativeProcess}

 The following describes the generative process for filling the $n^{th}$ word-position in document $d$ in the case of the STM \citep{roberts_structural_2013}: As in the case of conventional models, first a specific topic $z_{dn}$ is assigned, according to the topic distribution for that document $\theta$ through the process:

\begin{equation}
	z_{dn}|\theta_d \sim \textrm{Multinomial}(\theta_d).
\end{equation}

To incorporate the covariate values for that document, a topic-prevalence vector $\theta_d$ is drawn from a logistic-normal distribution:

\begin{equation}
	\theta_d|y_{d\gamma},\Sigma \sim \textrm{LogisticNormal}(\mu = y_{d\gamma}\Sigma),
\end{equation}

where $y_d\gamma$ lists the values of the metadata covariates for document $d$ and $\gamma$ relates these covariate values to the topic-prevalence. 

Conditional in the topic chosen ($z_{dn}$), a specific word $w_{dn}$, is selected from the overall corpus vocabulary $V$, using the following process:

\begin{equation}
	w_{dn}|z_{dn},\phi_{dkv} \sim \textrm{Multinomial}(\phi_{dk1},...,\phi_{dkV}),
\end{equation}

where the word probability $\phi_{dkv}$ is parameterized in terms of log-transformed rate deviations from the rates of a corpus-wide background distribution $m_v$ \citep{roberts_structural_2013}. The log-transformed rate deviations can then be specified by a collection of parameters $\lbrace \boldsymbol{\kappa} \rbrace$, where $\kappa^{(t)}$ is a $K$-by-$V$ matrix containing the log-transformed rate deviations for each topic $k$ and term $v$, over the baseline log-transformed rate for term $v$. This matrix is the same for all $A$ levels of covariates. To put it differently, $\kappa^{(t)}$ indicates the importance of the term $v$ given topic $k$ regardless of the covariates. Similarly, $\kappa^{(c)}$ is a $A$-by-$V$ matrix, indicating the importance of the term $v$ given the covariate level $c$ regardless of the topic. Finally, $\kappa^{(i)}$ is a $A$-by-$K$-by-$V$ matrix, collecting the covariate-topic effects:

\begin{equation}
	\phi_{dkv}|z_{dn}=\frac{\textrm{exp}(m_v+\kappa^{(t)}_{kv},\kappa^{(c)}_{y_dv}+\kappa^{(i)}_{y_dkv})}{\sum_v \textrm{exp}(m_v+\kappa^{(t)}_{kv},\kappa^{(c)}_{y_dv}+\kappa^{(i)}_{y_dkv})}.
\end{equation}

The STM maximizes the posterior likelihood that the observed data were generated by the above data-generating process using an iterative approximation-based variational expectation-maximization algorithm\footnote{A technical description of this maximization process can be found in \citet{roberts_model_2016}} available in R's stm package \citep{roberts_stm:_2016}. 

This process generates two posterior distribution parameters: 

\begin{enumerate}
	\item $\phi$ is a $K$-by-$V$ matrix (where $K=$ number of topics and $V=$ vocabulary or unique terms), where the entry $\phi_{kvc}$ can be interpreted as the probability of observing the $v$-th word in topic $k$ for the covariate level $c$ (the news website). 
	\item $\theta$ is a $D$-by-$V$ matrix (where $D=$ number of documents and $V=$ vocabulary or unique terms) of the document-topic distributions, where the entry $\theta_{dk}$ can be interpreted as the proportion of words in document $d$ which arise from topic $k$, or rather as the probability that document $d$ deals about topic $k$. 
\end{enumerate}


\subsection{Wordclouds}\label{apx_wordclouds}
\begin{figure}[H]
	\begin{center}
		\begin{subfigure}[normla]{0.43\textwidth}
			\includegraphics[width=\textwidth]{figs/wordcloud/DIEWELT.png}
			\caption{DIE WELT}
		\end{subfigure}
		\begin{subfigure}[normla]{0.43\textwidth}
			\includegraphics[width=\textwidth]{figs/wordcloud_FOCUSONLINE.png}
			\caption{FOCUS ONLINE}
		\end{subfigure}
		\begin{subfigure}[normla]{0.43\textwidth}
			\includegraphics[width=\textwidth]{figs/wordcloud_SPIEGELONLINE.png}
			\caption{SPIEGEL ONLINE}
		\end{subfigure}
		\begin{subfigure}[normla]{0.43\textwidth}
			\includegraphics[width=\textwidth]{figs/wordcloud_stern.png}
			\caption{Stern.de}
		\end{subfigure}
		\begin{subfigure}[normla]{0.43\textwidth}
			\includegraphics[width=\textwidth]{figs/wordcloud_ZEITONLINE.png}
			\caption{ZEIT ONLINE}
		\end{subfigure}
	\end{center}
\end{figure}

\subsection{Most frequent words}\label{apx_tf}
% 1, 2
\begin{figure}[H]
	\begin{center}
		\begin{subfigure}[normla]{0.49\textwidth}
			\includegraphics[width=\textwidth]{figs/plotquote1.png}
		\end{subfigure}
		\begin{subfigure}[normla]{0.49\textwidth}
			\includegraphics[width=\textwidth]{figs/plotquote2.png}
		\end{subfigure}
	\end{center}
\end{figure}
	
% 4, 10
\begin{figure}[H]
	\begin{center}
		\begin{subfigure}[normla]{0.49\textwidth}
			\includegraphics[width=\textwidth]{figs/plotquote4.png}
		\end{subfigure}
		\begin{subfigure}[normla]{0.49\textwidth}
			\includegraphics[width=\textwidth]{figs/plotquote10.png}
		\end{subfigure}
	\end{center}
\end{figure}

% 13, 17
\begin{figure}[H]
	\begin{center}
		\begin{subfigure}[normla]{0.49\textwidth}
			\includegraphics[width=\textwidth]{figs/plotquote13.png}
		\end{subfigure}
		\begin{subfigure}[normla]{0.49\textwidth}
			\includegraphics[width=\textwidth]{figs/plotquote17.png}
		\end{subfigure}
	\end{center}
\end{figure}

% 20, 23
\begin{figure}[H]
	\begin{center}
		\begin{subfigure}[normla]{0.49\textwidth}
			\includegraphics[width=\textwidth]{figs/plotquote20.png}
		\end{subfigure}
		\begin{subfigure}[normla]{0.49\textwidth}
			\includegraphics[width=\textwidth]{figs/plotquote23.png}
		\end{subfigure}
	\end{center}
\end{figure}

% 26, 27
\begin{figure}[H]
	\begin{center}
		\begin{subfigure}[normla]{0.49\textwidth}
			\includegraphics[width=\textwidth]{figs/plotquote26.png}
		\end{subfigure}
		\begin{subfigure}[normla]{0.49\textwidth}
			\includegraphics[width=\textwidth]{figs/plotquote27.png}
		\end{subfigure}
	\end{center}
\end{figure}

% 30, 32
\begin{figure}[H]
	\begin{center}
		\begin{subfigure}[normla]{0.49\textwidth}
			\includegraphics[width=\textwidth]{figs/plotquote30.png}
		\end{subfigure}
		\begin{subfigure}[normla]{0.49\textwidth}
			\includegraphics[width=\textwidth]{figs/plotquote32.png}
		\end{subfigure}
	\end{center}
\end{figure}

% 37, 46
\begin{figure}[H]
	\begin{center}
		\begin{subfigure}[normla]{0.49\textwidth}
			\includegraphics[width=\textwidth]{figs/plotquote37.png}
		\end{subfigure}
		\begin{subfigure}[normla]{0.49\textwidth}
			\includegraphics[width=\textwidth]{figs/plotquote46.png}
		\end{subfigure}
	\end{center}
\end{figure}

\subsection{Regression Results}\label{apx_coeff}
\begin{minipage}[t]{0.49\textwidth}
	% latex table generated in R 3.4.2 by xtable 1.8-2 package
% Tue May  8 10:07:43 2018
\begin{adjustbox}{width=\textwidth,totalheight=\textheight,keepaspectratio}
\begin{tabular}{llrrrr}
  \hline
topic\_name & parameter & Estimate & Std. Error & t value & p \\ 
  \hline
1: SPD, M.Schulz & (Intercept) & 0.036 & 0.003 & 11.460 & 0.000 \\ 
  1: SPD, M.Schulz & FOCUS ONLINE & -0.007 & 0.004 & -1.777 & 0.076 \\ 
  1: SPD, M.Schulz & SPIEGEL ONLINE & -0.008 & 0.004 & -1.965 & 0.049 \\ 
  1: SPD, M.Schulz & stern.de & -0.009 & 0.004 & -2.340 & 0.019 \\ 
  1: SPD, M.Schulz & Tagesschau.de & -0.015 & 0.005 & -3.311 & 0.001 \\ 
  1: SPD, M.Schulz & DIE WELT & -0.015 & 0.004 & -3.834 & 0.000 \\ 
  1: SPD, M.Schulz & ZEIT ONLINE & -0.012 & 0.004 & -2.686 & 0.007 \\ 
  2: B90/ Die Grünen & (Intercept) & 0.018 & 0.003 & 7.057 & 0.000 \\ 
  2: B90/ Die Grünen & FOCUS ONLINE & -0.005 & 0.003 & -1.515 & 0.130 \\ 
  2: B90/ Die Grünen & SPIEGEL ONLINE & 0.001 & 0.003 & 0.260 & 0.795 \\ 
  2: B90/ Die Grünen & stern.de & 0.005 & 0.003 & 1.689 & 0.091 \\ 
  2: B90/ Die Grünen & Tagesschau.de & -0.003 & 0.004 & -0.805 & 0.421 \\ 
  2: B90/ Die Grünen & DIE WELT & -0.004 & 0.003 & -1.160 & 0.246 \\ 
  2: B90/ Die Grünen & ZEIT ONLINE & 0.004 & 0.004 & 1.035 & 0.301 \\ 
  4: Great Coalition debates & (Intercept) & 0.064 & 0.005 & 13.210 & 0.000 \\ 
  4: Great Coalition debates & FOCUS ONLINE & -0.010 & 0.006 & -1.698 & 0.089 \\ 
  4: Great Coalition debates & SPIEGEL ONLINE & 0.003 & 0.006 & 0.487 & 0.626 \\ 
  4: Great Coalition debates & stern.de & -0.032 & 0.006 & -5.472 & 0.000 \\ 
  4: Great Coalition debates & Tagesschau.de & -0.005 & 0.006 & -0.841 & 0.401 \\ 
  4: Great Coalition debates & DIE WELT & -0.012 & 0.006 & -2.183 & 0.029 \\ 
  4: Great Coalition debates & ZEIT ONLINE & 0.002 & 0.006 & 0.400 & 0.689 \\ 
  10: M.Schulz, vs. A.Merkel & (Intercept) & 0.009 & 0.002 & 4.131 & 0.000 \\ 
  10: M.Schulz, vs. A.Merkel & FOCUS ONLINE & 0.001 & 0.003 & 0.192 & 0.848 \\ 
  10: M.Schulz, vs. A.Merkel & SPIEGEL ONLINE & 0.002 & 0.003 & 0.565 & 0.572 \\ 
  10: M.Schulz, vs. A.Merkel & stern.de & 0.007 & 0.003 & 2.860 & 0.004 \\ 
  10: M.Schulz, vs. A.Merkel & Tagesschau.de & 0.003 & 0.003 & 0.991 & 0.322 \\ 
  10: M.Schulz, vs. A.Merkel & DIE WELT & -0.003 & 0.003 & -1.117 & 0.264 \\ 
  10: M.Schulz, vs. A.Merkel & ZEIT ONLINE & 0.012 & 0.003 & 3.798 & 0.000 \\ 
  13: A.Merkel, election campaign & (Intercept) & 0.024 & 0.003 & 9.704 & 0.000 \\ 
  13: A.Merkel, election campaign & FOCUS ONLINE & 0.002 & 0.003 & 0.550 & 0.582 \\ 
  13: A.Merkel, election campaign & SPIEGEL ONLINE & 0.003 & 0.003 & 0.992 & 0.321 \\ 
  13: A.Merkel, election campaign & stern.de & 0.009 & 0.003 & 2.709 & 0.007 \\ 
  13: A.Merkel, election campaign & Tagesschau.de & -0.006 & 0.003 & -1.824 & 0.068 \\ 
  13: A.Merkel, election campaign & DIE WELT & 0.001 & 0.003 & 0.297 & 0.767 \\ 
  13: A.Merkel, election campaign & ZEIT ONLINE & 0.004 & 0.004 & 1.137 & 0.256 \\ 
  17: Debates within SPD & (Intercept) & 0.035 & 0.003 & 10.632 & 0.000 \\ 
  17: Debates within SPD & FOCUS ONLINE & -0.005 & 0.004 & -1.315 & 0.189 \\ 
  17: Debates within SPD & SPIEGEL ONLINE & -0.009 & 0.004 & -2.186 & 0.029 \\ 
  17: Debates within SPD & stern.de & -0.001 & 0.004 & -0.361 & 0.718 \\ 
  17: Debates within SPD & Tagesschau.de & -0.013 & 0.005 & -2.809 & 0.005 \\ 
  17: Debates within SPD & DIE WELT & -0.010 & 0.004 & -2.612 & 0.009 \\ 
  17: Debates within SPD & ZEIT ONLINE & -0.007 & 0.005 & -1.579 & 0.114 \\ 
  20: AfD, right-wing radicalism & (Intercept) & 0.017 & 0.003 & 5.704 & 0.000 \\ 
  20: AfD, right-wing radicalism & FOCUS ONLINE & -0.002 & 0.004 & -0.607 & 0.544 \\ 
  20: AfD, right-wing radicalism & SPIEGEL ONLINE & -0.003 & 0.004 & -0.724 & 0.469 \\ 
  20: AfD, right-wing radicalism & stern.de & -0.003 & 0.004 & -0.784 & 0.433 \\ 
  20: AfD, right-wing radicalism & Tagesschau.de & -0.000 & 0.004 & -0.033 & 0.973 \\ 
  20: AfD, right-wing radicalism & DIE WELT & 0.002 & 0.004 & 0.512 & 0.609 \\ 
  20: AfD, right-wing radicalism & ZEIT ONLINE & -0.002 & 0.004 & -0.538 & 0.590 \\ 
  22: SPD, stuffing debates & (Intercept) & 0.019 & 0.003 & 7.408 & 0.000 \\ 
  22: SPD, stuffing debates & FOCUS ONLINE & -0.004 & 0.003 & -1.170 & 0.242 \\ 
  22: SPD, stuffing debates & SPIEGEL ONLINE & 0.010 & 0.003 & 2.960 & 0.003 \\ 
  22: SPD, stuffing debates & stern.de & -0.006 & 0.003 & -1.943 & 0.052 \\ 
  22: SPD, stuffing debates & Tagesschau.de & -0.003 & 0.003 & -0.903 & 0.367 \\ 
  22: SPD, stuffing debates & DIE WELT & -0.004 & 0.003 & -1.308 & 0.191 \\ 
  22: SPD, stuffing debates & ZEIT ONLINE & 0.005 & 0.003 & 1.449 & 0.147 \\ 
   \hline
\end{tabular}
\end{adjustbox}

\end{minipage}
\begin{minipage}[t]{0.49\textwidth}
	% latex table generated in R 3.4.2 by xtable 1.8-2 package
% Tue May  8 10:07:44 2018
\begin{adjustbox}{width=\textwidth,totalheight=\textheight,keepaspectratio}
\begin{tabular}{llrrrr}
  \hline
topic\_name & parameter & Estimate & Std. Error & t value & p \\ 
  \hline
23: CSU, Söder \& Seehover & (Intercept) & 0.035 & 0.004 & 9.191 & 0.000 \\ 
  23: CSU, Söder \& Seehover & FOCUS ONLINE & -0.004 & 0.005 & -0.887 & 0.375 \\ 
  23: CSU, Söder \& Seehover & SPIEGEL ONLINE & 0.006 & 0.005 & 1.094 & 0.274 \\ 
  23: CSU, Söder \& Seehover & stern.de & -0.002 & 0.005 & -0.365 & 0.715 \\ 
  23: CSU, Söder \& Seehover & Tagesschau.de & -0.006 & 0.005 & -1.214 & 0.225 \\ 
  23: CSU, Söder \& Seehover & DIE WELT & -0.006 & 0.004 & -1.238 & 0.216 \\ 
  23: CSU, Söder \& Seehover & ZEIT ONLINE & 0.002 & 0.005 & 0.357 & 0.721 \\ 
  26: Jamaica fail, Reelections or GroKo? & (Intercept) & 0.035 & 0.003 & 11.775 & 0.000 \\ 
  26: Jamaica fail, Reelections or GroKo? & FOCUS ONLINE & -0.004 & 0.004 & -1.137 & 0.256 \\ 
  26: Jamaica fail, Reelections or GroKo? & SPIEGEL ONLINE & -0.004 & 0.004 & -1.076 & 0.282 \\ 
  26: Jamaica fail, Reelections or GroKo? & stern.de & -0.012 & 0.004 & -3.463 & 0.001 \\ 
  26: Jamaica fail, Reelections or GroKo? & Tagesschau.de & -0.010 & 0.004 & -2.584 & 0.010 \\ 
  26: Jamaica fail, Reelections or GroKo? & DIE WELT & -0.012 & 0.004 & -3.333 & 0.001 \\ 
  26: Jamaica fail, Reelections or GroKo? & ZEIT ONLINE & -0.005 & 0.004 & -1.356 & 0.175 \\ 
  27: Jamaica Coalition debates & (Intercept) & 0.066 & 0.005 & 13.865 & 0.000 \\ 
  27: Jamaica Coalition debates & FOCUS ONLINE & -0.029 & 0.006 & -5.152 & 0.000 \\ 
  27: Jamaica Coalition debates & SPIEGEL ONLINE & -0.009 & 0.006 & -1.438 & 0.150 \\ 
  27: Jamaica Coalition debates & stern.de & -0.028 & 0.006 & -5.088 & 0.000 \\ 
  27: Jamaica Coalition debates & Tagesschau.de & -0.018 & 0.006 & -2.939 & 0.003 \\ 
  27: Jamaica Coalition debates & DIE WELT & -0.008 & 0.005 & -1.416 & 0.157 \\ 
  27: Jamaica Coalition debates & ZEIT ONLINE & -0.003 & 0.007 & -0.460 & 0.645 \\ 
  30: AfD, F.Petry \& Meuthen & (Intercept) & 0.026 & 0.003 & 7.579 & 0.000 \\ 
  30: AfD, F.Petry \& Meuthen & FOCUS ONLINE & -0.005 & 0.004 & -1.168 & 0.243 \\ 
  30: AfD, F.Petry \& Meuthen & SPIEGEL ONLINE & 0.002 & 0.004 & 0.430 & 0.667 \\ 
  30: AfD, F.Petry \& Meuthen & stern.de & -0.004 & 0.004 & -0.896 & 0.370 \\ 
  30: AfD, F.Petry \& Meuthen & Tagesschau.de & -0.014 & 0.004 & -3.035 & 0.002 \\ 
  30: AfD, F.Petry \& Meuthen & DIE WELT & -0.011 & 0.004 & -2.812 & 0.005 \\ 
  30: AfD, F.Petry \& Meuthen & ZEIT ONLINE & 0.004 & 0.005 & 0.914 & 0.361 \\ 
  32: AfD, Gauland \& Weidel & (Intercept) & 0.023 & 0.004 & 6.308 & 0.000 \\ 
  32: AfD, Gauland \& Weidel & FOCUS ONLINE & 0.001 & 0.004 & 0.191 & 0.848 \\ 
  32: AfD, Gauland \& Weidel & SPIEGEL ONLINE & -0.001 & 0.005 & -0.123 & 0.902 \\ 
  32: AfD, Gauland \& Weidel & stern.de & 0.007 & 0.004 & 1.593 & 0.111 \\ 
  32: AfD, Gauland \& Weidel & Tagesschau.de & -0.006 & 0.005 & -1.221 & 0.222 \\ 
  32: AfD, Gauland \& Weidel & DIE WELT & 0.006 & 0.004 & 1.378 & 0.168 \\ 
  32: AfD, Gauland \& Weidel & ZEIT ONLINE & 0.006 & 0.005 & 1.198 & 0.231 \\ 
  37: AfD \& DIE LINKE in parliament & (Intercept) & 0.024 & 0.004 & 6.652 & 0.000 \\ 
  37: AfD \& DIE LINKE in parliament & FOCUS ONLINE & 0.005 & 0.004 & 1.221 & 0.222 \\ 
  37: AfD \& DIE LINKE in parliament & SPIEGEL ONLINE & 0.013 & 0.005 & 2.984 & 0.003 \\ 
  37: AfD \& DIE LINKE in parliament & stern.de & 0.001 & 0.004 & 0.252 & 0.801 \\ 
  37: AfD \& DIE LINKE in parliament & Tagesschau.de & 0.005 & 0.005 & 1.130 & 0.258 \\ 
  37: AfD \& DIE LINKE in parliament & DIE WELT & 0.002 & 0.004 & 0.483 & 0.629 \\ 
  37: AfD \& DIE LINKE in parliament & ZEIT ONLINE & 0.011 & 0.005 & 2.143 & 0.032 \\ 
  46: CDU & (Intercept) & 0.018 & 0.002 & 7.336 & 0.000 \\ 
  46: CDU & FOCUS ONLINE & -0.001 & 0.003 & -0.245 & 0.806 \\ 
  46: CDU & SPIEGEL ONLINE & 0.002 & 0.003 & 0.753 & 0.451 \\ 
  46: CDU & stern.de & -0.006 & 0.003 & -1.942 & 0.052 \\ 
  46: CDU & Tagesschau.de & -0.011 & 0.003 & -3.364 & 0.001 \\ 
  46: CDU & DIE WELT & -0.001 & 0.003 & -0.403 & 0.687 \\ 
  46: CDU & ZEIT ONLINE & -0.000 & 0.003 & -0.023 & 0.982 \\ 
   \hline
\end{tabular}
\end{adjustbox}

\end{minipage}

\subsection{Sentiment Values (monthly aggregated)}\label{apx_sentscore_monthly}

\begin{minipage}[t]{0.49\textwidth}
	% latex table generated in R 3.4.2 by xtable 1.8-2 package
% Tue May  8 10:36:37 2018
\begin{adjustbox}{width=\textwidth}
\begin{tabular}{llrr}
  \hline
date & topic & Number of Obs. & sentiment value * 1000 \\ 
  \hline
2017-06-01 & 1: SPD, M.Schulz &  51 & -0.04 \\ 
  2017-07-01 & 1: SPD, M.Schulz &  20 & -0.01 \\ 
  2017-08-01 & 1: SPD, M.Schulz &  49 & -0.02 \\ 
  2017-09-01 & 1: SPD, M.Schulz & 103 & -0.05 \\ 
  2017-10-01 & 1: SPD, M.Schulz &  42 & -0.01 \\ 
  2017-11-01 & 1: SPD, M.Schulz &  34 & -0.04 \\ 
  2017-12-01 & 1: SPD, M.Schulz &  24 & -0.04 \\ 
  2018-01-01 & 1: SPD, M.Schulz &  29 & -0.02 \\ 
  2018-02-01 & 1: SPD, M.Schulz &  47 & -0.18 \\ 
  2017-06-01 & 2: B90/ Die Grünen &  50 & -0.08 \\ 
  2017-07-01 & 2: B90/ Die Grünen &   8 & -0.00 \\ 
  2017-08-01 & 2: B90/ Die Grünen &   8 & -0.01 \\ 
  2017-09-01 & 2: B90/ Die Grünen &  51 & -0.01 \\ 
  2017-10-01 & 2: B90/ Die Grünen &  12 & -0.02 \\ 
  2017-11-01 & 2: B90/ Die Grünen &  22 & -0.02 \\ 
  2017-12-01 & 2: B90/ Die Grünen &  31 & -0.03 \\ 
  2018-01-01 & 2: B90/ Die Grünen &  59 & -0.06 \\ 
  2018-02-01 & 2: B90/ Die Grünen &   1 & -0.00 \\ 
  2017-06-01 & 4: Great Coalition debates &   1 & -0.00 \\ 
  2017-09-01 & 4: Great Coalition debates &   2 & -0.00 \\ 
  2017-10-01 & 4: Great Coalition debates &   1 & -0.00 \\ 
  2017-11-01 & 4: Great Coalition debates &  84 & -0.07 \\ 
  2017-12-01 & 4: Great Coalition debates & 219 & -0.15 \\ 
  2018-01-01 & 4: Great Coalition debates & 382 & -0.29 \\ 
  2018-02-01 & 4: Great Coalition debates & 119 & -0.03 \\ 
  2017-06-01 & 10: M.Schulz, vs. A.Merkel &  11 & -0.02 \\ 
  2017-07-01 & 10: M.Schulz, vs. A.Merkel &  50 & -0.06 \\ 
  2017-08-01 & 10: M.Schulz, vs. A.Merkel &  19 & -0.01 \\ 
  2017-09-01 & 10: M.Schulz, vs. A.Merkel &  56 & -0.02 \\ 
  2017-10-01 & 10: M.Schulz, vs. A.Merkel &   1 & -0.00 \\ 
  2017-11-01 & 10: M.Schulz, vs. A.Merkel &   4 & -0.00 \\ 
  2017-12-01 & 10: M.Schulz, vs. A.Merkel &   7 & -0.00 \\ 
  2018-01-01 & 10: M.Schulz, vs. A.Merkel &   5 & -0.01 \\ 
  2018-02-01 & 10: M.Schulz, vs. A.Merkel &   5 & -0.03 \\ 
  2017-06-01 & 13: A.Merkel, election campaign &  56 & -0.04 \\ 
  2017-07-01 & 13: A.Merkel, election campaign &  13 & -0.02 \\ 
  2017-08-01 & 13: A.Merkel, election campaign &  40 & -0.04 \\ 
  2017-09-01 & 13: A.Merkel, election campaign &  98 & -0.05 \\ 
  2017-10-01 & 13: A.Merkel, election campaign &   6 & -0.01 \\ 
  2017-11-01 & 13: A.Merkel, election campaign &  18 & -0.01 \\ 
  2017-12-01 & 13: A.Merkel, election campaign &  17 & -0.01 \\ 
  2018-01-01 & 13: A.Merkel, election campaign &   7 & -0.01 \\ 
  2018-02-01 & 13: A.Merkel, election campaign &  17 & -0.05 \\ 
  2017-06-01 & 17: Debates within SPD &  14 & -0.00 \\ 
  2017-07-01 & 17: Debates within SPD &   1 & -0.00 \\ 
  2017-08-01 & 17: Debates within SPD &   9 & -0.02 \\ 
  2017-09-01 & 17: Debates within SPD &  13 & -0.02 \\ 
  2017-10-01 & 17: Debates within SPD &  13 & -0.01 \\ 
  2017-11-01 & 17: Debates within SPD &  20 & -0.04 \\ 
  2017-12-01 & 17: Debates within SPD &  70 & -0.07 \\ 
  2018-01-01 & 17: Debates within SPD & 122 & -0.16 \\ 
  2018-02-01 & 17: Debates within SPD & 161 & -0.30 \\ 
  2017-06-01 & 20: AfD, right-wing radicalism &  19 & -0.05 \\ 
  2017-07-01 & 20: AfD, right-wing radicalism &  19 & -0.04 \\ 
  2017-08-01 & 20: AfD, right-wing radicalism &  58 & -0.19 \\ 
  2017-09-01 & 20: AfD, right-wing radicalism &  32 & -0.03 \\ 
  2017-10-01 & 20: AfD, right-wing radicalism &  31 & -0.06 \\ 
  2017-11-01 & 20: AfD, right-wing radicalism &  15 & -0.02 \\ 
  2017-12-01 & 20: AfD, right-wing radicalism &  21 & -0.02 \\ 
  2018-01-01 & 20: AfD, right-wing radicalism &  30 & -0.05 \\ 
  2018-02-01 & 20: AfD, right-wing radicalism &  15 & -0.08 \\ 
  2017-06-01 & 22: SPD, stuffing debates &   4 & -0.00 \\ 
  2017-07-01 & 22: SPD, stuffing debates &  11 & 0.01 \\ 
  2017-08-01 & 22: SPD, stuffing debates &   4 & -0.00 \\ 
  2017-09-01 & 22: SPD, stuffing debates &  43 & -0.04 \\ 
  2017-10-01 & 22: SPD, stuffing debates &  34 & -0.03 \\ 
   \hline
\end{tabular}
\end{adjustbox}

\end{minipage}
\begin{minipage}[t]{0.49\textwidth}
	% latex table generated in R 3.4.2 by xtable 1.8-2 package
% Tue May  8 10:36:37 2018
\begin{adjustbox}{width=\textwidth}
\begin{tabular}{llrr}
  \hline
date & topic & Number of Obs. & sentiment value * 1000 \\ 
  \hline
2017-11-01 & 22: SPD, stuffing debates &  19 & -0.01 \\ 
  2017-12-01 & 22: SPD, stuffing debates &  19 & -0.01 \\ 
  2018-01-01 & 22: SPD, stuffing debates &  22 & -0.02 \\ 
  2018-02-01 & 22: SPD, stuffing debates &  48 & -0.09 \\ 
  2017-06-01 & 23: CSU, Söder \& Seehover &   1 & -0.00 \\ 
  2017-07-01 & 23: CSU, Söder \& Seehover &  25 & -0.00 \\ 
  2017-08-01 & 23: CSU, Söder \& Seehover &  13 & -0.02 \\ 
  2017-09-01 & 23: CSU, Söder \& Seehover &  70 & -0.06 \\ 
  2017-10-01 & 23: CSU, Söder \& Seehover & 125 & -0.15 \\ 
  2017-11-01 & 23: CSU, Söder \& Seehover & 105 & -0.12 \\ 
  2017-12-01 & 23: CSU, Söder \& Seehover &  83 & -0.00 \\ 
  2018-01-01 & 23: CSU, Söder \& Seehover &  16 & -0.02 \\ 
  2018-02-01 & 23: CSU, Söder \& Seehover &  23 & -0.01 \\ 
  2017-06-01 & 26: Jamaica fail, Reelections or GroKo? &   3 & -0.00 \\ 
  2017-07-01 & 26: Jamaica fail, Reelections or GroKo? &   3 & -0.00 \\ 
  2017-08-01 & 26: Jamaica fail, Reelections or GroKo? &   5 & -0.01 \\ 
  2017-09-01 & 26: Jamaica fail, Reelections or GroKo? &  11 & -0.01 \\ 
  2017-10-01 & 26: Jamaica fail, Reelections or GroKo? &   7 & -0.01 \\ 
  2017-11-01 & 26: Jamaica fail, Reelections or GroKo? & 230 & -0.28 \\ 
  2017-12-01 & 26: Jamaica fail, Reelections or GroKo? &  29 & -0.04 \\ 
  2018-01-01 & 26: Jamaica fail, Reelections or GroKo? &  17 & -0.03 \\ 
  2018-02-01 & 26: Jamaica fail, Reelections or GroKo? &  10 & -0.02 \\ 
  2017-06-01 & 27: Jamaica Coalition debates &   7 & -0.00 \\ 
  2017-07-01 & 27: Jamaica Coalition debates &   8 & -0.00 \\ 
  2017-08-01 & 27: Jamaica Coalition debates &   8 & -0.02 \\ 
  2017-09-01 & 27: Jamaica Coalition debates &  69 & -0.06 \\ 
  2017-10-01 & 27: Jamaica Coalition debates & 233 & 0.01 \\ 
  2017-11-01 & 27: Jamaica Coalition debates & 418 & -0.36 \\ 
  2017-12-01 & 27: Jamaica Coalition debates &  46 & -0.06 \\ 
  2018-01-01 & 27: Jamaica Coalition debates &   7 & -0.01 \\ 
  2018-02-01 & 27: Jamaica Coalition debates &   2 & -0.02 \\ 
  2017-06-01 & 30: AfD, F.Petry \& Meuthen &  16 & -0.02 \\ 
  2017-07-01 & 30: AfD, F.Petry \& Meuthen &  27 & -0.07 \\ 
  2017-08-01 & 30: AfD, F.Petry \& Meuthen &  27 & -0.07 \\ 
  2017-09-01 & 30: AfD, F.Petry \& Meuthen &  88 & -0.05 \\ 
  2017-10-01 & 30: AfD, F.Petry \& Meuthen &  55 & -0.10 \\ 
  2017-11-01 & 30: AfD, F.Petry \& Meuthen &  31 & -0.01 \\ 
  2017-12-01 & 30: AfD, F.Petry \& Meuthen &  27 & -0.05 \\ 
  2018-01-01 & 30: AfD, F.Petry \& Meuthen &   6 & -0.01 \\ 
  2018-02-01 & 30: AfD, F.Petry \& Meuthen &   5 & -0.01 \\ 
  2017-06-01 & 32: AfD, Gauland \& Weidel &  20 & -0.04 \\ 
  2017-07-01 & 32: AfD, Gauland \& Weidel &  17 & -0.03 \\ 
  2017-08-01 & 32: AfD, Gauland \& Weidel &  34 & -0.08 \\ 
  2017-09-01 & 32: AfD, Gauland \& Weidel & 167 & -0.14 \\ 
  2017-10-01 & 32: AfD, Gauland \& Weidel &  24 & -0.02 \\ 
  2017-11-01 & 32: AfD, Gauland \& Weidel &  24 & -0.03 \\ 
  2017-12-01 & 32: AfD, Gauland \& Weidel &  71 & -0.04 \\ 
  2018-01-01 & 32: AfD, Gauland \& Weidel &  29 & -0.06 \\ 
  2018-02-01 & 32: AfD, Gauland \& Weidel &  29 & -0.09 \\ 
  2017-06-01 & 37: AfD \& DIE LINKE in parliament &  10 & -0.02 \\ 
  2017-08-01 & 37: AfD \& DIE LINKE in parliament &   4 & -0.00 \\ 
  2017-09-01 & 37: AfD \& DIE LINKE in parliament &  65 & -0.03 \\ 
  2017-10-01 & 37: AfD \& DIE LINKE in parliament & 138 & -0.15 \\ 
  2017-11-01 & 37: AfD \& DIE LINKE in parliament &  34 & -0.03 \\ 
  2017-12-01 & 37: AfD \& DIE LINKE in parliament &  37 & -0.06 \\ 
  2018-01-01 & 37: AfD \& DIE LINKE in parliament &  95 & -0.15 \\ 
  2018-02-01 & 37: AfD \& DIE LINKE in parliament &  16 & -0.04 \\ 
  2017-06-01 & 46: CDU &  13 & -0.01 \\ 
  2017-07-01 & 46: CDU &  16 & -0.03 \\ 
  2017-08-01 & 46: CDU &  31 & -0.05 \\ 
  2017-09-01 & 46: CDU &  17 & -0.02 \\ 
  2017-10-01 & 46: CDU &  39 & -0.07 \\ 
  2017-11-01 & 46: CDU &  18 & -0.01 \\ 
  2017-12-01 & 46: CDU &  20 & -0.02 \\ 
  2018-01-01 & 46: CDU &  11 & -0.01 \\ 
  2018-02-01 & 46: CDU &  30 & -0.06 \\ 
   \hline
\end{tabular}
\end{adjustbox}

\end{minipage}

\subsection{Sentiment Values (aggregated by news website)}\label{apx_sentscore_site}

\begin{minipage}[t]{0.49\textwidth}
	% latex table generated in R 3.4.2 by xtable 1.8-2 package
% Tue May  8 10:40:30 2018
\begin{adjustbox}{width=\textwidth}
\begin{tabular}{llrr}
  \hline
News website & topic & Number of Obs. & sentiment value * 1000 \\ 
  \hline
Bild.de & 1: SPD, M.Schulz &  44 & -0.08 \\ 
  Bild.de & 10: M.Schulz, vs. A.Merkel &  10 & -0.01 \\ 
  Bild.de & 13: A.Merkel, election campaign &  18 & -0.01 \\ 
  Bild.de & 17: Debates within SPD &  46 & -0.06 \\ 
  Bild.de & 2: B90/ Die Grünen &  21 & -0.00 \\ 
  Bild.de & 20: AfD, right-wing radicalism &  22 & -0.06 \\ 
  Bild.de & 22: SPD, stuffing debates &  19 & -0.02 \\ 
  Bild.de & 23: CSU, Söder \& Seehover &  38 & -0.04 \\ 
  Bild.de & 26: Jamaica fail, Reelections or GroKo? &  43 & -0.09 \\ 
  Bild.de & 27: Jamaica Coalition debates &  96 & -0.11 \\ 
  Bild.de & 30: AfD, F.Petry \& Meuthen &  30 & -0.04 \\ 
  Bild.de & 32: AfD, Gauland \& Weidel &  28 & -0.04 \\ 
  Bild.de & 37: AfD \& DIE LINKE in parliament &  30 & -0.05 \\ 
  Bild.de & 4: Great Coalition debates &  83 & -0.08 \\ 
  Bild.de & 46: CDU &  17 & -0.03 \\ 
  DIE WELT & 1: SPD, M.Schulz &  71 & -0.02 \\ 
  DIE WELT & 10: M.Schulz, vs. A.Merkel &  19 & -0.02 \\ 
  DIE WELT & 13: A.Merkel, election campaign &  50 & -0.02 \\ 
  DIE WELT & 17: Debates within SPD &  70 & -0.05 \\ 
  DIE WELT & 2: B90/ Die Grünen &  34 & -0.03 \\ 
  DIE WELT & 20: AfD, right-wing radicalism &  62 & -0.07 \\ 
  DIE WELT & 22: SPD, stuffing debates &  34 & -0.01 \\ 
  DIE WELT & 23: CSU, Söder \& Seehover &  78 & -0.04 \\ 
  DIE WELT & 26: Jamaica fail, Reelections or GroKo? &  54 & -0.04 \\ 
  DIE WELT & 27: Jamaica Coalition debates & 188 & -0.06 \\ 
  DIE WELT & 30: AfD, F.Petry \& Meuthen &  40 & -0.03 \\ 
  DIE WELT & 32: AfD, Gauland \& Weidel & 108 & -0.07 \\ 
  DIE WELT & 37: AfD \& DIE LINKE in parliament &  75 & -0.05 \\ 
  DIE WELT & 4: Great Coalition debates & 166 & -0.07 \\ 
  DIE WELT & 46: CDU &  49 & -0.04 \\ 
  FOCUS ONLINE & 1: SPD, M.Schulz &  84 & -0.06 \\ 
  FOCUS ONLINE & 10: M.Schulz, vs. A.Merkel &  23 & -0.03 \\ 
  FOCUS ONLINE & 13: A.Merkel, election campaign &  50 & -0.02 \\ 
  FOCUS ONLINE & 17: Debates within SPD &  78 & -0.08 \\ 
  FOCUS ONLINE & 2: B90/ Die Grünen &  33 & -0.01 \\ 
  FOCUS ONLINE & 20: AfD, right-wing radicalism &  43 & -0.05 \\ 
  FOCUS ONLINE & 22: SPD, stuffing debates &  32 & -0.02 \\ 
  FOCUS ONLINE & 23: CSU, Söder \& Seehover &  74 & -0.05 \\ 
  FOCUS ONLINE & 26: Jamaica fail, Reelections or GroKo? &  56 & -0.06 \\ 
  FOCUS ONLINE & 27: Jamaica Coalition debates & 103 & -0.09 \\ 
  FOCUS ONLINE & 30: AfD, F.Petry \& Meuthen &  48 & -0.02 \\ 
  FOCUS ONLINE & 32: AfD, Gauland \& Weidel &  68 & -0.07 \\ 
  FOCUS ONLINE & 37: AfD \& DIE LINKE in parliament &  73 & -0.06 \\ 
  FOCUS ONLINE & 4: Great Coalition debates & 144 & -0.06 \\ 
  FOCUS ONLINE & 46: CDU &  47 & -0.03 \\ 
  SPIEGEL ONLINE & 1: SPD, M.Schulz &  46 & -0.05 \\ 
  SPIEGEL ONLINE & 10: M.Schulz, vs. A.Merkel &  18 & -0.02 \\ 
  SPIEGEL ONLINE & 13: A.Merkel, election campaign &  30 & -0.02 \\ 
  SPIEGEL ONLINE & 17: Debates within SPD &  44 & -0.05 \\ 
  SPIEGEL ONLINE & 2: B90/ Die Grünen &  34 & -0.04 \\ 
  SPIEGEL ONLINE & 20: AfD, right-wing radicalism &  25 & -0.06 \\ 
  SPIEGEL ONLINE & 22: SPD, stuffing debates &  46 & -0.04 \\ 
   \hline
\end{tabular}
\end{adjustbox}

\end{minipage}
\begin{minipage}[t]{0.49\textwidth}
	% latex table generated in R 3.5.1 by xtable 1.8-2 package
% Wed Aug 22 17:14:44 2018
\begin{table}[ht]
\centering
\begin{tabular}{llrr}
  \hline
News website & topic & Number of Obs. & sentiment value * 1000 \\ 
  \hline
SPIEGEL ONLINE & 23: CSU, Söder \& Seehover &  74 & -0.06 \\ 
  SPIEGEL ONLINE & 26: Jamaica fail, Reelections or GroKo? &  60 & -0.08 \\ 
  SPIEGEL ONLINE & 27: Jamaica Coalition debates & 119 & -0.08 \\ 
  SPIEGEL ONLINE & 30: AfD, F.Petry \& Meuthen &  66 & -0.09 \\ 
  SPIEGEL ONLINE & 32: AfD, Gauland \& Weidel &  35 & -0.04 \\ 
  SPIEGEL ONLINE & 37: AfD \& DIE LINKE in parliament &  74 & -0.09 \\ 
  SPIEGEL ONLINE & 4: Great Coalition debates & 115 & -0.04 \\ 
  stern.de & 1: SPD, M.Schulz &  65 & -0.00 \\ 
  stern.de & 10: M.Schulz, vs. A.Merkel &  64 & -0.02 \\ 
  stern.de & 13: A.Merkel, election campaign &  56 & -0.03 \\ 
  stern.de & 17: Personnel debates SPD (Schulz vs. Gabriel) &  84 & -0.07 \\ 
  stern.de & 2: B90/ Die Grünen & 142 & -0.05 \\ 
  stern.de & 22: Personnel debates SPD (Nahles vs. Schwesig) &  34 & -0.02 \\ 
  stern.de & 23: CSU, Söder \& Seehover &  86 & -0.04 \\ 
  stern.de & 26: Jamaica fail, Reelections or GroKo? &  76 & -0.06 \\ 
  stern.de & 27: Jamaica Coalition debates & 110 & -0.02 \\ 
  stern.de & 30: AfD, F.Petry \& Meuthen &  83 & -0.07 \\ 
  stern.de & 32: AfD, Gauland \& Weidel &  73 & -0.05 \\ 
  stern.de & 37: AfD \& DIE LINKE in parliament &  63 & -0.04 \\ 
  stern.de & 4: Great Coalition debates & 110 & -0.04 \\ 
  Tagesschau.de & 1: SPD, M.Schulz &  12 & -0.01 \\ 
  Tagesschau.de & 10: M.Schulz, vs. A.Merkel &  27 & -0.02 \\ 
  Tagesschau.de & 13: A.Merkel, election campaign &  17 & -0.03 \\ 
  Tagesschau.de & 17: Personnel debates SPD (Schulz vs. Gabriel) &  58 & -0.06 \\ 
  Tagesschau.de & 2: B90/ Die Grünen &  36 & -0.00 \\ 
  Tagesschau.de & 22: Personnel debates SPD (Nahles vs. Schwesig) &  21 & 0.01 \\ 
  Tagesschau.de & 23: CSU, Söder \& Seehover &  45 & -0.06 \\ 
  Tagesschau.de & 26: Jamaica fail, Reelections or GroKo? &  42 & -0.02 \\ 
  Tagesschau.de & 27: Jamaica Coalition debates &  82 & -0.03 \\ 
  Tagesschau.de & 30: AfD, F.Petry \& Meuthen &  27 & -0.04 \\ 
  Tagesschau.de & 32: AfD, Gauland \& Weidel &  18 & -0.03 \\ 
  Tagesschau.de & 37: AfD \& DIE LINKE in parliament &  50 & -0.04 \\ 
  Tagesschau.de & 4: Great Coalition debates & 101 & -0.05 \\ 
  ZEIT ONLINE & 1: SPD, M.Schulz &  31 & 0.00 \\ 
  ZEIT ONLINE & 10: M.Schulz, vs. A.Merkel &  40 & -0.04 \\ 
  ZEIT ONLINE & 13: A.Merkel, election campaign &  28 & -0.03 \\ 
  ZEIT ONLINE & 17: Personnel debates SPD (Schulz vs. Gabriel) &  57 & -0.09 \\ 
  ZEIT ONLINE & 2: B90/ Die Grünen &  68 & -0.07 \\ 
  ZEIT ONLINE & 22: Personnel debates SPD (Nahles vs. Schwesig) &  32 & -0.03 \\ 
  ZEIT ONLINE & 23: CSU, Söder \& Seehover &  43 & -0.03 \\ 
  ZEIT ONLINE & 26: Jamaica fail, Reelections or GroKo? &  38 & -0.09 \\ 
  ZEIT ONLINE & 27: Jamaica Coalition debates &  94 & -0.08 \\ 
  ZEIT ONLINE & 30: AfD, F.Petry \& Meuthen &  62 & -0.11 \\ 
  ZEIT ONLINE & 32: AfD, Gauland \& Weidel &  36 & -0.05 \\ 
  ZEIT ONLINE & 37: AfD \& DIE LINKE in parliament &  52 & -0.04 \\ 
   \hline
\end{tabular}
\end{table}

\end{minipage}

\subsection{Cross Correlation Coefficient (lag 0)}\label{apx_ccf}
% latex table generated in R 3.4.2 by xtable 1.8-2 package
% Wed May  9 00:35:34 2018
\begin{table}[ht]
\centering
\begin{tabular}{rlrrrrrr}
  \hline
 & Var1 & AfD & FDP & Grüne & Linke & SPD & Union \\ 
  \hline
1 & 1: SPD, M.Schulz & -0.636 & 0.072 & -0.438 & -0.693 & 0.752 & 0.199 \\ 
  2 & 10: M.Schulz, vs. A.Merkel & 0.438 & 0.566 & 0.505 & 0.341 & -0.229 & -0.758 \\ 
  3 & 13: A.Merkel, election campaign & 0.098 & 0.350 & 0.440 & -0.060 & 0.076 & -0.507 \\ 
  4 & 17: Debates within SPD & -0.770 & 0.075 & -0.676 & -0.754 & 0.841 & 0.382 \\ 
  5 & 2: B90/ Die Grünen & 0.253 & -0.092 & 0.185 & 0.480 & -0.329 & -0.048 \\ 
  6 & 20: AfD, right-wing radicalism & 0.225 & 0.369 & 0.224 & 0.037 & -0.157 & -0.319 \\ 
  7 & 22: SPD, stuffing debates & -0.710 & -0.112 & -0.465 & -0.797 & 0.877 & 0.302 \\ 
  8 & 23: CSU, Söder \& Seehover & -0.169 & -0.897 & -0.249 & -0.283 & 0.145 & 0.565 \\ 
  9 & 26: Jamaica fail, Reelections or GroKo? & -0.251 & -0.501 & -0.388 & -0.258 & 0.047 & 0.537 \\ 
  10 & 27: Jamaica Coalition debates & -0.197 & -0.439 & -0.266 & -0.222 & 0.000 & 0.436 \\ 
  11 & 30: AfD, F.Petry \& Meuthen & 0.410 & -0.032 & 0.351 & 0.296 & -0.348 & -0.224 \\ 
  12 & 32: AfD, Gauland \& Weidel & -0.220 & 0.430 & 0.221 & -0.315 & 0.336 & -0.296 \\ 
  13 & 37: AfD \& DIE LINKE in parliament & -0.217 & -0.521 & -0.418 & -0.003 & 0.238 & 0.552 \\ 
  14 & 4: Great Coalition debates & -0.165 & 0.336 & -0.361 & 0.292 & -0.126 & 0.228 \\ 
  15 & 46: CDU & -0.083 & -0.066 & -0.108 & -0.286 & 0.285 & -0.007 \\ 
   \hline
\end{tabular}
\caption{Cross Correlation at lag 0} 
\label{t_ccf}
\end{table}


\end{document}
